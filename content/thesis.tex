%\section{PhD Thesis}
\section{Th\`ese de doctorat}
\cvitem{Titre}{\emph{Deformable object segmentation in ultra-sound images.}}
\cvitem{Supervisors}{\href{http://eia.udg.edu/~joanm/index-uk.html}{Joan Mart\'i} and \href{http://scholar.google.fr/citations?user=tNttgvEAAAAJ&hl=en&oi=sra}{Prof. Fabrice Meriaudeau}}
\cvitem{Description}{
Sa th\`ese est consacr\'ee \`a la segmentation automatique des l\'esions mammaires dans les images \'echographiques car cette t\`ache est essentielle pour le d\'eveloppement de syst\`emes robustes de diagnostic assist\'e par ordinateur (CAD) appliqu\'es \`a cet organe et \`a cette modalit\'e d'image. La strat\'egie de segmentation propos\'ee divise les images en r\'egions significatives appel\'ees super-pixels et les \'etiquette en utilisant un cadre de minimisation qui prend en compte la formation et la r\'egularisation.
}
%% This thesis is devoted to automatic segmentation of breast lesions in ultrasound images, since this task is key for the development of robust Computer Aided Diagnosis (CAD) systems applied to this organ and image modality. The proposed segmentation strategy divides the images into meaningful regions called super-pixels and labels them using a minimization framework that takes into account training and regularization.
