\documentclass[a4paper,10pt]{article}
% Importation de packages divers
   \NeedsTeXFormat{LaTeX2e} 
   \usepackage[usenames]{color}
   \usepackage[paper=a4paper,textwidth=160mm,twosideshift=0pt,hmargin=2.5cm, vmargin=1.5cm]{geometry}
   \usepackage{fancyhdr}
   \usepackage{lastpage}  
   \usepackage{wasysym}
   \usepackage{ marvosym }
   \usepackage{titlesec}
\titleformat{\section}[block]{\Large\bfseries\filcenter}{}{1em}{}
   \usepackage{float}
\restylefloat{table}
    % pour l'affichage du n� de la derni\`{e}re page.
   \usepackage[latin1]{inputenc}        % utilisation des caract\`{e}rres 8 bits en Unix (codage ISO 8859-1)
 %  \usepackage{floatflt,multirow}       % pour l'utilisation des figure ``noy\'{e}e'' dans le texte
   \usepackage[francais]{babel}         % Utilisation du fran�ais (nom des sections, c\'{e}sure, ponctuation, ...)
  % \usepackage{amsmath,amsthm,amssymb}  % Utilisation de certains packages de AMS (cf. belles \'{e}quations)
   \usepackage{fontenc}
   \usepackage[cyr]{aeguill}
   \usepackage{endnotes}                % Pour l'utilisation des notes en fin de documents
   \usepackage{verbatim}                % Pour l'insertion de fichier en mode verbatim
   \usepackage[pdftitle={Fichier realise a partir de base_lettre.tex},  % apparition ds les propri\'{e}t\'{e}s du doc
               pdfauthor={Joan Massich},
               pdfsubject={Qualification},
               pdfkeywords={Qualification},
	       colorlinks=true,
	       linkcolor=webdarkblue, 
	       filecolor=webblue, 
	       urlcolor=webdarkblue,
	       citecolor=webgreen]{hyperref}     % pour l'utilisation des liens http,...
   \usepackage{portland}		% pour l'utilisation de \portrait et de \landscape sur une page
\pagestyle{fancy}
%opening
\title{\LARGE{CANDIDATURE A LA QUALIFICATION
AUX FONCTIONS DE MAITRE DE CONFERENCES} \\
	\LARGE{Section CNU: 27\up{\`{e}me}} }
\author{\textbf{Joan \textsc{Massich}}}
\date{16/12/2016}
\begin{document}

\maketitle
\thispagestyle{fancy}
  \lhead{Joan \textsc{Massich}}
  \rhead{}
  \chead{Dossier de candidature}
  \cfoot{}
  \vspace*{3cm}
  \section*{Etat civil}
 \begin{table}[H]
 \center
 \begin{tabular}{l l} 
   Nom: & \textbf{Massich} \\ \\ [-2ex]
   Pr\'{e}nom: & \textbf{Joan} \\ \\ [-2ex]
   Date et lieu de naissance: & \textbf{15 Mars 1984 � Palafrugell (Espagne)} \\ \\ [-2ex]
   Nationalit\'{e}: & \textbf{Espagnol} \\ \\ [-2ex]
   Situation de famille: & \textbf{C\'elibataire} \\ \\ [-2ex]
   Coordonn\'{e}es professionnelles & \textbf{Le2i - UMR CNRS 6306} \\
   & \textbf{12 Rue de la Fonderie} \\
   & \textbf{71200 Le Creusot} \\
   & \phone \textbf{+33 6 01 20 16 68} \\
   & \Email \textbf{mailsik@gmail.com} \\ \\ [-2ex]
   Coordonn\'{e}es personnelles: & \textbf{Joan Massich} \\
   & \textbf{30bis Rue du bois} \\
   & \textbf{La Chapelle du bas} \\
   & \textbf{89290 Venoy} \\
   & \Mobilefone \textbf{+33 6 01 20 16 68} \\
   & \Email \textbf{mailsik@gmail.com} 
\end{tabular}
 \end{table}
 
 %Vision omnidirectionnelle $\&$ hybride, Structure-From-Motion, Suivi-visuel, Auto-calibrage, Robotique mobile
 \section*{R\'{e}sum\'{e}}
\begin{table}[h]
\begin{tabular}{llll}
\multirow{}{}{Publications:}  & \textbf{Revues:}              & \textbf{2}   & 2 en auteur correspondant          \\
                                & \textbf{Conf\'{e}rences}          & \textbf{23}   & 6 en 1\textsuperscript{er} auteur  \\
                                & \textbf{Rapport technique}            & \textbf{2}   &                           \\ \\ [-2ex]
 
\multirow{}{}{Enseignements:} & \textbf{Architecture des syst\`{e}mes}  & \textbf{20h} & DUT IQ 1\textsuperscript{\`{e}me} ann\'{e}e \\
                              & \textbf{Conception Orient\'{e}e Object} & \textbf{72h} & DUT IQ 1\textsuperscript{\`{e}me} ann\'{e}e \\
                                & \textbf{Programmation Web Orient\'{e} Client}  & \textbf{44h} & DUT IQ 2\textsuperscript{\`{e}me} ann\'{e}e  \\
                                & \textbf{Programmation mobile} & \textbf{28h} & DUT IQ 2\textsuperscript{\`{e}me} ann\'{e}e  \\ 
                                & \textbf{Total \'{e}quivalent TD} & \textbf{117.3 h} & \\

\end{tabular}
\end{table}
Mots-cl\'{e}s:
10:14 Web s\'{e}mantique,
20:22 Graphes,
30:33 Logique et fondaments de la programmation,
50:53 Mod\'{e}lisqtion et simulation,
70:75 M\'{e}todes de programmation et paradigmes,
80:81 Apprentissage,
90:91 Analyse d'images,
90:93 Perception et vision par ordinateur,
C0:C1 syst\`{e}mes embarqu\'{e}s.

\end{document}
            



