\documentclass[a4paper,10pt]{article}
% Importation de packages divers
   \NeedsTeXFormat{LaTeX2e}
   \usepackage[usenames]{color}
   \usepackage[paper=a4paper,textwidth=160mm,hmargin=2.5cm, vmargin=1.5cm]{geometry}
   \usepackage{fancyhdr}
   \usepackage{lastpage}
   \usepackage[usenames,dvipsnames]{xcolor}
   \usepackage{float}
   \usepackage{subfigure}
   \definecolor{light-gray}{gray}{0.95}
   \usepackage{titlesec}
   \usepackage{epsf,graphicx}
   \usepackage[style=numeric,backend=biber,maxcitenames=10,
   mincitenames=10, maxbibnames=99, minbibnames=99, sorting=none
   ]{biblatex}
   %\usepackage{biblatex}
%\bibliography{bib1}

   %% \usepackage[
   %%   backend=biber,
   %%   style=alphabetic,
   %%   sorting=ynt
   %% ]{biblatex}

\addbibresource{bib1.bib}
\usepackage{epstopdf}
   \titleformat*{\section}{\centering\normalfont\Large\bfseries}

\restylefloat{table}
    % pour l'affichage du n� de la derni\`{e}re page.
   \usepackage[latin1]{inputenc}        % utilisation des caract\`{e}rres 8 bits en Unix (codage ISO 8859-1)
 %  \usepackage{floatflt,multirow}       % pour l'utilisation des
 %  figure ``noy\'{e}e'' dans le texte
   \usepackage{multirow}
   \usepackage[francais]{babel}         % Utilisation du fran�ais (nom des sections, c\'{e}sure, ponctuation, ...)
   \usepackage{amsmath,amsthm,amssymb}  % Utilisation de certains packages de AMS (cf. belles \'{e}quations)
   \usepackage{fontenc}
   \usepackage[cyr]{aeguill}
   \usepackage{endnotes}                % Pour l'utilisation des notes en fin de documents
   \usepackage{verbatim}                % Pour l'insertion de fichier en mode verbatim
   % \usepackage[pdftitle={Dossier de Qualifiation},  % apparition ds les propri\'{e}t\'{e}s du doc
   %             pdfauthor={Guillaume Lemaitre},
   %             pdfsubject={Qualification},
   %             pdfkeywords={Simplicite},
   %             colorlinks=true,
   %             linkcolor=webdarkblue,
   %             filecolor=webblue,
   %             urlcolor=webdarkblue,
   %             citecolor=webgreen]{hyperref}     % pour l'utilisation des liens http,...
   \usepackage{portland}		% pour l'utilisation de \portrait et de \landscape sur une page
%\pagestyle{fancy}
%opening

   \usepackage{booktabs}

\begin{document}

\section{Activit\'{e}s p\'{e}dagogiques}
  \vspace{-0.5cm}
  \begin{center}
\underline{\hspace{8cm}}
\end{center}

\subsection{R\'{e}sum\'{e} des enseignements}

Vous pourrez trouver dans le tableau~\ref{tab:enseignement}, un r\'{e}sum\'{e}
des enseignements effectu\'{e}s durant ma th\`{e}se. Ref\'erez-vous \`{a} la
section~\ref{sec:details} pour les d\'{e}tails concernant chaque
enseignement.

\begin{table}[h]
  \centering
  {\renewcommand{\arraystretch}{1.2}
    \caption{R\'{e}capitulatif des enseignements effectu\'{e}s}
    \begin{tabular}{llcccc}
      \toprule
      \textbf{Ann\'{e}e} &
      \textbf{Enseignement} &
      \textbf{Niveau}          & \multicolumn{3}{c}{\textbf{Volume}} \\ \cmidrule(lr){4-6}
      & & & CM & TD & TP           \\ \midrule
      \multirow{3}{*}{2015/2016}
      & Architecture des syst\`{e}mes  & DUT IQ 1\textsuperscript{\`{e}me} ann\'{e}e & & & 20h \\
      & Conception Orient\'{e}e Object & DUT IQ 1\textsuperscript{\`{e}me} ann\'{e}e & & & 72h \\
      & Programmation Web Orient\'{e} Client & DUT IQ 2\textsuperscript{\`{e}me} ann\'{e}e & & 12h & 32h \\
      & Programmation mobile & DUT IQ 2\textsuperscript{\`{e}me} ann\'{e}e & & 12h & 16h \\

      \midrule
      \textbf{Total} & & &  & \textbf{24h} & \textbf{140h} \\ %117.3 equivalent en TD
      \bottomrule
    \end{tabular}}
  \label{tab:enseignement}
\end{table}

\subsection{D\'{e}tails des enseignements}\label{sec:details}

\begin{description}
\item[Conception orient\'{e}e objets]

L'objectif de ces travaux pratiques est d'initier les \'{e}tudiants de premi\`{e}re
ann\'{e}e DUT informatique \`{a} la mod\'{e}lisation orient\'{e}e objet. les principes de la
programmation orient\'{e}e-objet tels que : l'encapsulation, l'h\'{e}ritage et le
polymorphisme ont \'{e}t\'{e} abord\'{e}s.
\\
De fa�on plus sp\'{e}cifique, ce cours permet \`{a} l'\'{e}tudiant de:
\begin{itemize}
\item Ma\^{i}triser une suite de mod\'{e}lisation UML comme \emph{Visual Paradigm}
\item D\'{e}couvrir le d\'{e}veloppment du logiciel dans un \'{e}quipe en utilisant
des syst\'{e}mes de version contr\^{o}le et agile
\item Ma\^{i}triser \`{a} programmer dans le paradigme orient\'{e}-objet avec le langage Java.\\
\end{itemize}
% \vspace{2pt}

\item[Architecture des systems]

L'objectif de ces travaux pratiques \'{e}tait de permettre aux \'{e}tudiants de premi\`{e}re
ann\'{e}e DUT informatique ayant d\'{e}j\`{a} des connaissances de base en programmation C
de ma\^{i}triser la programmation a bas niveau, comprendre la conception et commandament des syst\`{e}mes
d'entr\'{e}e-sortie dans un environnement de micro-controler simul\'{e}.

% \vspace{2pt}
\item[Programmation mobile]
Les 12h de travaux dirig\'{e}s, et les 16h de travaux pratiques ont permis aux
\'{e}tudiants de deuxi\`{e}me ann\'{e}e DUT informatique de d\'{e}velopper des applications
mobiles sous Android afin d'apprendre \`{a} cr\'{e}er une application native pour
t\'{e}lephone ou tablette.

% \vspace{2pt}
\item[Clients web riches]

Les 12h de travaux dirig\'{e}s et les 32h de travaux pratiques ont \'{e}t\'{e} dispens\'{e} aux
\'{e}tudiants de deuxi\`{e}me ann\'{e}e DUT informatique. L'objectif \'{e}tait que les matr\^{i}ce
des concepts n\'{e}cessaire pour d\'{e}velopper:
\begin{itemize}
\item un jeu web complet.
\item le logiciel ncesaire pour construire l'interface web a connecter avec le
  project ERP d\'{e}velopp\'{e} dans un autre module de cours.
\end{itemize}

\end{description}

\subsection{Supervision de projet}

Durant ma p\'{e}riode postdoctorale, j'ai l'occasion de sup\'{e}rvis\'{e} des \'{e}tudiants de
master dans le laboratoire et de collaborer avec la tasc de superviser les
\'{e}studiants de doctorat.

\subsection{Membre actif du project SALEIE}

Durant ma p\'{e}riode postdoctorale, j'ai l'occasion de travailler dans le cadre
du project H2020:SALEIE (Strategic Alignment of Electrical and Information Engineering
in European Higher Education Institutions).
L\`{a} je fais partie de la revision des programmes d'enseignement sup\'{e}rieur
europ\'{e}en et j'ai particip\'{e} \`{a} la cr\'{e}ation de directirces por la
definition des programmes dans le domaine de technologies de l'information et de
la communication (TIC). Un r\'{e}sum\'{e} a \'{e}t\'{e} publi\'{e} dans~\cite{ligusova2014reflections}.

%% \subsection{Responsabilit\'{e} d'unit\'{e} d'enseignement}

%% J'ai donc assur\'{e} la coordination du module d'introduction au
%% traitement de l'image dans le Master Vibot.
\newpage
\section{Activit\'{e}s de recherches}
  \vspace{-0.5cm}
  \begin{center}
\underline{\hspace{8cm}}
\end{center}

\subsection{Doctorat}

\begin{itemize}
%\item Discipline : \textbf{Instrumentation et Informatique de l'Image}
\item Titre : \textbf{Segmentation d�bjects d\'{e}formables en imagerie ultrasonore}
\item Institutions : Universit\'{e} de Bourgogne au laboratoire Le2i
(Laboratoire d'Electronique, Informatique et Image) / Universitat de Girona \`{a} Institut VICOROB
\item P\'{e}riode : Octobre 2009 \`{a} D\'{e}cembre 2013 
\item Soutenue le : 4 d\'{e}cembre 2013 \item mention : Tr\`{e}s Honorable \item Directeur de th\`{e}se : \textbf{Fabrice Meriaudeau}, Professeur \`{a} l'Universit\'{e} de Bourgogne
\item Codirecteur de th\`{e}se : \textbf{Joan Mart\'{i}}, Professeur \`{a} l'Universitat de Girona

\item Jury de th\`{e}se :
\end{itemize}
\begin{table}[h]
\centering
\resizebox{0.9\textwidth}{!}{
\begin{tabular}{lllll}
\toprule
Denis Friboulet      & Professeur             & Institut national des sciences appliqu\'{e}es de Lyon         & Pr\'{e}sident du jury  & CNU 61 \\ 
Robert Mart\'{i}     & Maitre de conf\'{e}rence   & Universitat de Girona (Vicorob)                           & Co-directeur       & - \\ 
Fabrice Meriaudeau   & Professeur             & Universit\'{e} de Bourgogne (Le2i)                            & Directeur de th\`{e}se & CNU 61 \\ 
% Francesco Tortorella & Professeur             & Universit\`{a} degli Studi di Cassino e del Lazio Meridionale & Directeur de th\`{e}se & - \\ 
Francesco Tortorella & Professeur             & Universit\`{a} degli Studi di Cassino & Directeur de th\`{e}se & - \\ 
\bottomrule
\end{tabular}}
\end{table}


\subsubsection{R\'{e}sum\'{e} de th\`{e}se}
Le cancer du sein est la cause principale de mortalit\'{e} par cancer chez les
femmes. Bien que la Mammographie Num\'{e}rique (MN) reste la r\'{e}f\'{e}rence
pour les m\'{e}thodes d'examen existantes, l'imagerie ultrasonore a prouv\'{e}
son efficacit\'{e} en tant que modalit\'{e} complementaire, et on estime qu'elle
pourait \'{e}viter 65 \`{a} 85\% des biopsies prescrites. Cependant, les images
ultrasonores sont difficilement interpr\'{e}tables, c'est pour cela que la
communaut\'{e} m\'{e}dicale a mis au point un lexique commun r\'{e}duisant les
incoh\'{e}rences entre radiologues. Une telle pratique est \'{e}norm\'{e}ment
couteuse en temps.

Les syst\`{e}mes de diagnostic assist\'{e} par ordinateur (DAO) ont \'{e}t\'{e} d\'{e}velopp\'{e}s afin
d'aider les radiologues dans la prise d\'{e}cision concernant les l\'{e}sions d\'{e}tect\'{e}es.
Cependant, ces syst\`{e}mes ne prennent pas en compte le lexique d\'{e}velopp\'{e} par ces
derniers, ce qui rend leurs utilisations compliqu\'{e}s.

Mes travaux de th\`{e}se ont eu pour but de concevoir un DAO compatible avec le
lexique mise en place par les m\'{e}decins. Une analyse du processus de segmentation
est effectu\'{e}e et une nouvelle m\'{e}thode automatique de segmentation sur des images
ultrasons (US) est propos\'{e}e.

\subsubsection{Les contributions}
\begin{description}
\item[Base de donn'{e}es publique d'images ultrasonores du sein]
Cr'{e}er un ensemble de donn'{e}es public d'images d''{e}chographie du sein avec des
d'{e}limitations de l'{e}sion et des annotations radiologues.
\item [Proposition de deux nouvelles m\'{e}thodologies]
  En plus d'une r\'{e}vision approfondie des m\'{e}thodologies
  existantes~\cite{massich2013phd}, j'ai propos\'{e}:
  \begin{itemize}
  \item La premi\'{e}re est bas\'{e}e sur la propagation d'un front d'onde en
  utilitsant un modele Gaussiane~\cite{massich2010lesion,massich2011seed}.
  \item Le second, utilis'{e} un framework de minimisation bas'{e} sur des coupes de graphe et
une repr'{e}sentation des images en
superpixels~\cite{massich2012seed,massich2014sift,massich2015miccai}.
  \end{itemize}

\end{description}

\subsection{Post-doctorat}

I did a postdoc research at medical imaging group at Le2i where simultaneously
we worked on several projects, including Skin cancer, prostate cancer, breast
cancer and OCT. My work as a postdoctorale researcher involved guiding the phd
students and working with them to advance the projects.

\begin{description}
\item[Classification automatique d'Ed\`{e}me Maculaire Diab\'{e}tique (EMD) sur
  imagerie TCO.]
  This two years project was based on the collaboration with the singapour eye
  research institute. Within this project we delivered two classification
  frameworks. The former corresponds to a supervised method
  \cite{lemaitre2015miccai, lemaitre2016classification}, wheras the later
  consists of an semi-supervised method~\cite{sidibe2017anomaly}.
  In this regard, a testing benchmark and a review comparing our methods with
  state of the art was published in~\cite{massich2016classification}.

\item[D'{e}veloppement de syst\`{e}mes DAO pour des applications en imagerie m'{e}dicale]
  In order to develop CAD systems for different medical applications, I
  contributed to standardize a common framework that allowed us to generalize the
  CAD systems developed for OCT and breast to skin and prostate
  cancer CAD systems.

  This work resulted in the following publications:~\cite{}.  

\end{description}

Besides medical imaging I also worked on several projects where the laboratorie
was contracted by an industrial partner. One of these projects lead to a join
publication with the non conventional imaging group at Le2i.

% \subsubsection{Les contributions}
% \begin{description}
% \item[Plateforme pour partager des benchmarks et des bases de donn\'{e}es
%   d'imagerie m\'{e}dicale] Cette platforme se trouve \`{a} \url{http://i2cvb.github.io}.

% \item[Extension des m\'{e}thodes propos\'{e}es dans~\cite{massich2013phd}]
%   proposant une nouvelle fonction de r\'{e}gularization et en incorporant de nouveaux
% descripteurs qui ont am'{e}lior'{e} nos r'{e}sultats pr'{e}c'{e}dents~\cite{massich2015miccai}.
% \item[]
% \end{description}

% \subsection{Travaux de recherche}

\subsection{Travaux de recherche}

Mes travaux de recherche se focalisent principalement sur des m\'{e}thodes
d'apprentissage statistiques et automatiques pur EMD et cancer du sein.

En parall�le de mes travaux de recherche, j'ai 
travaill\'{e} avec mes coll�gues sur d'autres probl\'{e}matiques de recherche
telsque les probl�mes de dataset d\'{e}s\'{e}quilibr\'{e}, de CADs d\'{e}di\'{e}s � la
d\'{e}tection de cancer de la prostate et du melanome.

J'ai \'{e}galement travaill\'{e} sur divers projets de recherche p\'{e}dagogique tels que SALARI\'{E} et Playful Coding.


\subsection{Perspective de recherche}

J'ai eu l'occasion d'exceller dans le domaine de l'apprentissage
statistique et automatique, du traitement d'images et de l'imagerie
non conventionnelle.
Ces techniques ont \'{e}t\'{e} sp\'{e}cifiquement appliqu\'{e}es au domaine de
l'imagerie m\'{e}dicale, l'imagerie non conventionelle.


\section{Autres activit\'{e}s}
  \vspace{-0.5cm}
  \begin{center}
\underline{\hspace{8cm}}
\end{center}


\subsection{Organisation d'\'{e}v\`{e}nements scientifiques}
J'ai \'{e}t\'{e} co-organis\'{e} la deuxi\`{e}me \'{e}dition du Doctoral Day 2015,
organis\'{e} au Creusot.

J'ai \'{e}galement particip\'{e} \`{a} l'organisation \`{a} la semaine d'int\'{e}gration,
le Vibot Day ainsi que la remise des dipl\^{o}mes du Master Erasmus Mundus
Vibot.

J'ai aussi cr�� un groupe de travail dans le laboratoire Le2i qui se r�unit
r�guli�rement � analyser des travaux scientifiques dans la litt\'{e}rature.

\subsection{Relecture d'articles scientifiques}
J'ai effectu\'{e} des relectures pour des revues scientifiques et de
conf\'{e}rences internationales.

\newpage

\section{Publications}
  \vspace{-0.5cm}
  \begin{center}
\underline{\hspace{8cm}}
\end{center}

Toutes mes revues publi\'{e}es sont toutes r\'{e}f\'{e}renc\'{e}es JCR.

\nocite{*}


%\printbibliography[type=article,title={Revues internationales}]
%\printbibliography[type=inproceedings,title={Conf\'{e}rences internationales}]

\printbibliography[keyword=journal,title={Revues internationales}]
\printbibliography[keyword=conference,title={Conf\'{e}rences internationales}]
\printbibliography[keyword=bookchapter,title={Chapitre du livre}]
\printbibliography[keyword=other,title={Divers}]

\newpage

\section{Annexes}
  \vspace{-0.5cm}
  \begin{center}
\underline{\hspace{8cm}}
\end{center}

Les documents suivants sont joints \`{a} ce dossier en annexe:
\begin{itemize}
\item
\item Attestation et recommandation de C\'edric Demonceaux, Professeur, responsable du site
  du Creusot - Le2i.
\item Attestation et recommandation de Sylvain Rampeck, Ma\^{i}tre de conf\'{e}rence, chef du d\'{e}partement informatique.
\item Recommandation de Joan Mart\'i, professeur \`{a} l'universitat de Girona.
\item Rapport de th\`{e}se confidenciel.
\item Rapport de th\`{e}se confidenciel.
\item Attestion de r\'{e}ussite au diplome de th\`{e}se.
\item 2 Publications en tant que premier auteur.
\end{itemize}

\end{document}
